\pagenumbering{arabic}
\section{Introduction and background}
\label{sec:introduction}

\subsection{Background}

作为生命的基本单位,人类对细胞和组织的研究从未停止。生物组织切片作为直接观察细胞形态和结构的重要手段,对于生物医学研究和临床诊断至关重要。一份完整且可用的组织切片对于研究人员和医生来说非常重要,因为它提供了关于细胞结构、组织形态和病理变化的重要信息。在此过程中,切片的质量至关重要。

传统的手动切片方法耗时且易于变化,因此自动切片机的出现为这些问题提供了解决方案。对于不同的生物组织,不同的切割参数可以产生不同的结果,既有积极的,也有消极的。因此,为了提高生物切片的利用率和增加高质量标本的产量,确定特定组织的最佳切割参数仍然是一个目标。

机器学习和深度学习在计算机视觉和图像处理领域取得了显著的成功。机器学习被定义为一系列可以自动检测数据中的模式的方法,这些模式随后用于预测未来的结果或做出决策 \cite{1.1}。在本文中,我们整合了先进的图像分析和机器学习技术,以识别切片质量,然后评估不同切割参数下组织样本的质量。

\subsection{Introduction}

\textbf{Project Overview}

本项目旨在优化生物组织切片机的切割参数,这些设备在生物医学研究和临床诊断中起着关键的作用。目标是通过确定最佳的切片条件,提高组织样本准备的精度和效率。通过收集在各种切割参数下的组织样本,并进行后续的手动图像分类,本研究采用深度学习技术来分析和预测最有效的切割参数。这项工作不仅有望提高组织样本的显微检查质量,而且有助于简化实验室工作流程,从而推动生物和医学科学的发展。


\textbf{Objectives:}

\begin{enumerate} 
    \item 收集在不同参数下切片的组织样本的全面数据集。 
    \item 采用人工图像分类来分类这些样本的质量和特性。 
    \item 开发和训练一个能够评估组织样本质量的深度学习模型。 
    \item 使用模型的洞察来确定组织切片机的最佳切割参数。 
    \item 通过实证测试和改进来验证模型的预测。 
\end{enumerate}

\subsection{Structure of the Report}

该项目分为几个章节,每个章节都旨在系统地探讨研究背景、方法、实验工作、结果展示、讨论和结论,以及项目管理、可持续性和健康安全的考虑:

\textbf{引言和背景} - 本章概述了项目的目标、目的和结构安排。简要介绍了研究的动机和必要性,以及采用的技术协议和标准。

\textbf{文献综述} - 对生物组织切片、图像分类和深度学习在生物样本准备中的应用进行了深入的讨论。本节将当前的研究置于现有研究的背景中。

\textbf{方法和理论} - 对实验方法、理论框架和数据收集和处理的具体计划进行了详细描述。

\textbf{实验工作/分析调查/设计} - 详细描述了实验设计、实施和分析调查。详细阐述了为实现项目目标所采用的策略和方法。

\textbf{实验或分析结果/最终构建产品描述} - 本章展示了实验数据、分析结果或最终设计产品,详细描述了实验或设计的结果。

\textbf{讨论和结论} - 分析了结果,讨论了它们的科学意义和实际价值。本章还提供了研究的结论,并提出了未来研究的可能方向。

\textbf{项目管理、可持续性和健康安全考虑} - 讨论了项目管理策略、可持续性问题和健康安全措施,以确保研究工作的高效和安全进行。

\textbf{参考文献} - 列出了所有参考文献,支持研究并为研究提供了基础。


\textbf{Assumptions and Technical Specifications}

该项目基于几个关键的假设和技术协议,包括:

\begin{enumerate} 
    \item 不同批次的组织样本的性质一致。 
    \item 生物组织切片机和成像设备的可靠性和准确性。 
    \item 深度学习模型在解释复杂的生物图像数据方面的充分性。 
\end{enumerate}

关于组织切片机设置、图像分类标准和深度学习架构的技术规格在方法和理论部分详细说明。

