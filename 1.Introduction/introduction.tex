\pagenumbering{arabic}
\section{Introduction and background}
\label{sec:introduction}


\subsection{Introduction}

\subsubsection{Project Overview}

This project addresses the critical task of optimizing the cutting parameters of a biological tissue slicer, an essential instrument in biomedical research and clinical diagnostics. The aim is to enhance the precision and efficiency of tissue sample preparation by identifying the optimal slicing conditions. Through the collection of tissue samples under various cutting parameters and subsequent artificial image classification, this study employs deep learning techniques to analyze and predict the most effective slicing parameters. This endeavor not only promises to improve the quality of tissue samples for microscopic examination but also to streamline the workflow in laboratories, thereby contributing to the advancement of biological and medical sciences.
% 这个项目解决了生物组织切片仪的关键任务,这是生物医学研究和临床诊断中的一个重要仪器。其目标是通过确定最佳切片条件来提高组织样本制备的精确性和效率。通过收集在不同切割参数下的组织样本,并进行人工图像分类,本研究采用深度学习技术来分析和预测最有效的切片参数。这个努力不仅有望改善显微镜检查的组织样本质量,还有望简化实验室的工作流程,从而促进生物和医学科学的进步。
\subsubsection{Objectives}

\begin{enumerate}
    \item Collect a comprehensive dataset of tissue samples sliced under different parameters.
    \item Employ artificial image classification to categorize the quality and characteristics of these samples.
    \item Develop and train a deep learning model capable of assessing tissue sample quality.
    \item Use the model's insights to determine the optimal cutting parameters for the tissue slicer.
    \item Validate the model's predictions through empirical testing and refinement.
\end{enumerate}

% \item 收集在不同参数下切割的组织样本的全面数据集。
% \item 使用人工图像分类来对这些样本的质量和特征进行分类。
% \item 开发和训练一个能够评估组织样本质量的深度学习模型。
% \item 利用模型的见解来确定组织切片仪的最佳切割参数。
% \item 通过实证测试和改进来验证模型的预测结果。
% \end{enumerate}

\subsubsection{Structure of the Report}

This project is organized into the following chapters, each designed to systematically explore the research background, methodologies, experimental work, results presentation, discussions and conclusions, as well as considerations for project management, sustainability, and health and safety:

\textbf{Introduction and Background} - This chapter outlines the project's objectives, goals, and structural arrangement. It provides a brief introduction to the motivation and necessity for the research, along with the technical protocols and specifications adopted.

\textbf{Literature Review} - An in-depth discussion on the use of biological tissue slicers, image classification, and deep learning in the preparation of biological samples. This section positions the current study within the context of existing research.

\textbf{Methodology and Theory} - Detailed descriptions of the experimental methods, theoretical frameworks, and the specific plans for data collection and processing are presented here.

\textbf{Experimental Work/Analytical Investigation/Design} - Describes the detailed steps of experimental design, implementation, and analytical investigation. It elaborates on the strategies and methods adopted to achieve the project's objectives.

\textbf{Presentation of Experimental or Analytical Results/Descriptions of Final Constructed Product} - This chapter showcases the experimental data, analysis results, or the final design product, providing detailed accounts of the experimental or design outcomes.

\textbf{Discussion and Conclusions} - The results are analyzed, and their scientific significance and practical value are discussed. This chapter also offers the research conclusions and suggests potential directions for future studies.

\textbf{Project Management, Consideration of Sustainability and Health and Safety} - Discusses strategies for project management, sustainability issues, and health and safety measures to ensure the research work is conducted efficiently and safely.

\textbf{References} - Lists all the bibliographic materials cited, supporting the research and providing the basis for the study.
% 本项目分为以下几个章节,每个章节都旨在系统地探索研究背景、方法论、实验工作、结果展示、讨论和结论,以及项目管理、可持续性和健康安全方面的考虑:

% \textbf{引言和背景} - 本章概述了项目的目标、目标和结构安排。它简要介绍了研究的动机和必要性,以及采用的技术协议和规范。

% \textbf{文献综述} - 对生物组织切片、图像分类和深度学习在生物样本制备中的应用进行深入讨论。本节将当前研究定位于现有研究的背景下。

% \textbf{方法和理论} - 详细描述了实验方法、理论框架以及数据收集和处理的具体计划。

% \textbf{实验工作/分析调查/设计} - 描述了实验设计、实施和分析调查的详细步骤。详细阐述了实现项目目标所采用的策略和方法。

% \textbf{实验或分析结果展示/最终构建产品描述} - 本章展示了实验数据、分析结果或最终设计产品,详细描述了实验或设计结果。

% \textbf{讨论和结论} - 对结果进行分析,讨论其科学意义和实际价值。本章还提供研究结论,并提出未来研究的潜在方向。

% \textbf{项目管理、可持续性和健康安全考虑} - 讨论项目管理策略、可持续性问题和健康安全措施,以确保研究工作的高效和安全进行。

% \textbf{参考文献} - 列出所有引用的文献资料,支持研究并为研究提供基础。

\subsubsection{Assumptions and Technical Specifications}

The project is based on several key assumptions and technical protocols, which are:

\begin{enumerate}
    \item The consistency in tissue sample properties across different batches.
    \item The reliability and precision of the biological tissue slicer and imaging equipment.
    \item The adequacy of the deep learning model in interpreting complex biological image data.
\end{enumerate}
Technical specifications regarding the tissue slicer settings, image classification criteria, and deep learning architecture are detailed in \textbf{Methodology and Theory}.
% 该项目基于几个关键假设和技术协议,包括:

% \begin{enumerate}
%     \item 不同批次之间组织样本性质的一致性。
%     \item 生物组织切片仪和成像设备的可靠性和精确性。
%     \item 深度学习模型在解释复杂生物图像数据方面的充分性。
% \end{enumerate}
% 有关组织切片仪设置、图像分类标准和深度学习架构的技术规格详见方法和理论。

\subsection{Background}

\subsubsection{Importance of Tissue Sample Quality}

High-quality tissue samples are pivotal for accurate diagnosis and research. The quality of a tissue sample can significantly affect the results of histological analysis, making the optimization of slicing parameters a crucial endeavor.

\subsubsection{Advancements in Image Classification and Deep Learning}

Recent advancements in image classification and deep learning have opened new avenues for automating and enhancing the analysis of biological samples. By leveraging these technologies, it is possible to achieve greater accuracy and efficiency in identifying optimal tissue slicing parameters.

\subsubsection{Gap in Current Research}

While there have been significant strides in both biological sample preparation and computational analysis, a gap remains in integrating these approaches to optimize tissue slicing parameters. This project aims to bridge this gap by developing a predictive model that can guide the adjustment of slicing conditions for optimal outcomes.

% \subsection{背景}

% \subsubsection{组织样本质量的重要性}

% 高质量的组织样本对于准确的诊断和研究至关重要。组织样本的质量可以显著影响组织学分析的结果,因此优化切片参数是一项关键的工作。

% \subsubsection{图像分类和深度学习的进展}

% 图像分类和深度学习的最新进展为自动化和增强生物样本分析开辟了新的途径。通过利用这些技术,可以在识别最佳组织切片参数方面实现更高的准确性和效率。

% \subsubsection{当前研究中的差距}

% 尽管在生物样本制备和计算分析方面取得了重大进展,但在整合这些方法以优化组织切片参数方面仍存在差距。本项目旨在通过开发一个预测模型来填补这一差距,该模型可以指导调整切片条件以获得最佳结果。

\section{Literature review}

This literature review examines the convergence of technologies in biological tissue slicing, with a particular focus on the application of image classification and deep learning to optimize slicing parameters. It aims to highlight significant advancements, identify gaps in current methodologies, and set the groundwork for the proposed project.

% 这篇文献综述探讨了生物组织切片中技术的融合,特别关注图像分类和深度学习在优化切片参数方面的应用。它旨在突出重要的进展,确定当前方法学中存在的差距,并为拟议的项目奠定基础。
\subsection{切片机与显微镜的选择}
%组织切片和图像获取的技术背景   这里写切片机和显微镜的技术参数和采集手册及方法


近年来,随着科技的发展,自动切片机的出现能够显著简化切片操作和提高切片质量。

M在《Improved reproducibility in preparing precision-cut liver tissue slices》一文中提出,使用新型徕卡振动刀片切片机可以提高大鼠、小鼠和人体组织切片的准确性和重现性。-----------



在本次实验中,使用epredia提供的HM355S机器进行切片处理。该机器是一款热门的用于生物组织切片研究的设备,有不少实验和论文都使用了这款设备进行切片处理。

Elzbieta Klimuszko 使用过HM355S机器,以牙齿作为标本进行切片操作,探究牙釉质中的钙镁含量。-------------
https://link.springer.com/article/10.1007/s10266-018-0353-6


Andelko Hrzenjak使用HM355S机器,对病变的子宫内膜组织进行切片操作,研究子宫内膜癌的发生机制。-------------
https://www.sciencedirect.com/science/article/pii/S1525157810605685

同样,对于显微镜的选择也是至关重要的。在本次实验中,使用了来自Keyence公司的VHX7000显微镜进行图像采集。他不仅能采集生物组织切片的图像(小鼠前列腺细胞
https://www.frontiersin.org/journals/oncology/articles/10.3389/fonc.2022.943846/full
),


还能采集无机物(如陶瓷,玻璃)的表面图像。
https://www.sciencedirect.com/science/article/abs/pii/S010956412300204X
https://www.taylorfrancis.com/chapters/edit/10.1201/9781003023555-130/looking-foundations-structural-glass-digital-microscope-veer

实验中将使用HM355s切片机和VHX7000显微镜进行切片和图像采集。

% 希望通过自动化操作过程可可靠地控制切片的质量,增加高质量标本的产量,减少次品标本的数量。然而,不同组织样本的最佳切割参数各不相同。确定特定组织的最佳切割参数仍然是一个挑战。

% \subsection{Image Classification for Tissue Analysis}

% %用于组织分析的图像分类   这里写图像分类的技术背景和方法


\subsection{关于切片组织的深度学习}

%深度学习在生物医学应用中的应用   这里写深度学习的技术背景和方法

深度学习技术在生物医学领域的应用已经取得了显著进展。深度学习模型在图像分类、目标检测和分割等任务中表现出色,为生物医学实验室的研究和诊断提供了强大的工具。

Lorena Guachi-Guachi 提出了一种使用cnn网络对组织切片进行识别并进行修整。

应用于切片术的卷积神经网络:识别石蜡包埋组织块的修剪末端切割程序
https://www.sciencedirect.com/science/article/abs/pii/S0952197623011478


在Biomedical Texture Analysis一书中,Vincent Andrearczyk提出了一种专门用于纹理分析的cnn架构,相比其它传统架构能够显著提高生物组织的分类准确性。

第 4 章-纹理分析中的深度学习及其在组织图像分类中的应用
https://www.sciencedirect.com/science/article/abs/pii/B9780128121337000041


Yan Xu提出,从大型自然图像数据库 ImageNet 训练的 CNN 中提取的特征能够转移到组织病理学图像中,这为我们实现迁移学习提供了一种可行的思路。

通过深度卷积激活特征进行大规模组织病理学图像分类、分割和可视化
https://link.springer.com/article/10.1186/s12859-017-1685-x

根据以上文献,深度学习技术在组织切片的图像分类和分析中具有广泛的应用前景。通过利用深度学习模型,可以实现对组织样本的高效识别和分类,为优化切片参数提供有力支持。


% \subsection{Integration of Deep Learning for Optimizing Tissue Slicing Parameters}

% %整合深度学习以优化组织切片参数   这里写深度学习在优化切片参数中的应用


\FloatBarrier % Now figures cannot float above section title


%% 草稿
% 历史上,组织切片主要依赖于手工技术,这既耗时又容易产生变异。近年来,自动切片机的发展旨在解决这些问题。然而,通过整合先进的图像分析和机器学习技术来优化切片参数仍然是一个挑战,很少有人涉及到这个领域。

% 生物医学成像领域在图像分类方面取得了显著的进展,特别是引入机器学习和深度学习模型。这些技术在高精度识别和分类组织特征方面显示出潜力,为自动评估组织切片质量提供了可能的途径。

% 深度学习因其处理和分析复杂数据集的能力而越来越受到认可,从而提高了生物医学实验室的效率。值得注意的是,一些研究已经开始探索深度学习在优化生物医学设备参数方面的潜力,尽管在组织切片优化方面的研究仍然有限。

% 尽管组织切片、图像分类和深度学习各自取得了显著进展,但在整合它们以优化组织切片参数的研究仍然很少。这为本项目提供了一个独特的机会,通过开发一个综合模型,将这些技术协同应用,为该领域做出贡献。

% 尽管深度学习和图像分类在生物医学应用中具有潜在的能力,但它们在优化组织切片参数方面的应用仍未得到充分探索。本项目旨在填补这一空白,通过开发一个基于深度学习的模型,预测最佳切片条件,从而提高组织样本在研究和诊断中的质量。


% 文献综述强调了通过整合深度学习和图像分类技术改进组织切片的重要潜力。通过解决已确定的研究差距,本项目有望对组织样本制备的效率和准确性做出有意义的贡献,对生物医学研究和诊断具有广泛的影响。
