\section{Literature review}

This literature review explores the integration of technologies in biological tissue sectioning, with a particular focus on the application of image classification and deep learning in optimizing slicing parameters. It aims to highlight significant advancements, identify gaps in current methodologies, and lay the groundwork for the proposed project.

% 这篇文献综述探讨了生物组织切片中技术的融合,特别关注图像分类和深度学习在优化切片参数方面的应用。它旨在突出重要的进展,确定当前方法学中存在的差距,并为拟议的项目奠定基础。
% \subsection{切片机与显微镜的选择}
\subsection{Microtome and Microscope Selection}

In recent years, the advent of automatic microtomes has significantly simplified the sectioning process and improved the quality of sections.

Zimmermann, in the article "Improved reproducibility in preparing precision-cut liver tissue slices," advocates for the use of the new Leica vibratome to enhance the accuracy and reproducibility of tissue sections from rats, mice, and human tissues \cite{LR.1}.

In this experiment, the HM355S microtome provided by Epredia is used for sectioning. This machine is a popular device for biological tissue sectioning research, and many experiments and papers have utilized this equipment for sectioning.

Elzbieta Klimuszko has used the HM355S microtome for sectioning teeth to investigate the calcium and magnesium content in dental enamel \cite{LR.2}.

Andelko Hrzenjak also used the HM355S microtome for sectioning pathological endometrial tissues to study the mechanisms of endometrial carcinoma development \cite{LR.3}.

Similarly, the choice of microscope is crucial. In this experiment, the VHX7000 microscope from Keyence is used for image acquisition. It is capable of capturing images of biological tissue sections (e.g., mouse prostate cells \cite{LR.4}),
as well as inorganic materials (such as ceramics \cite{LR.5}, glass \cite{LR.6}).

The experiments will employ the HM355s microtome and VHX7000 microscope for sectioning and image acquisition. This setup ensures that both equipment selection and technological application are optimally aligned to enhance the precision and efficiency of the tissue sectioning process, supporting the overall goals of the research project.

% 近年来,随着科技的发展,自动切片机的出现能够显著简化切片操作和提高切片质量。

% Zimmermann在《Improved reproducibility in preparing precision-cut liver tissue slices》一文中提出,使用新型徕卡振动刀片切片机可以提高大鼠、小鼠和人体组织切片的准确性和重现性。\cite{LR.1}


% 在本次实验中,使用epredia提供的HM355S机器进行切片处理。该机器是一款热门的用于生物组织切片研究的设备,有不少实验和论文都使用了这款设备进行切片处理。

% Elzbieta Klimuszko 使用过HM355S机器,以牙齿作为标本进行切片操作,探究牙釉质中的钙镁含量。\cite{LR.2}


% Andelko Hrzenjak使用HM355S机器,对病变的子宫内膜组织进行切片操作,研究子宫内膜癌的发生机制。\cite{LR.3}

% 同样,对于显微镜的选择也是至关重要的。在本次实验中,使用了来自Keyence公司的VHX7000显微镜进行图像采集。他不仅能采集生物组织切片的图像(小鼠前列腺细胞
% \cite{LR.4}
% ),
% 还能采集无机物(如陶瓷\cite{LR.5},玻璃\cite{LR.6})的表面图像。


% 实验中将使用HM355s切片机和VHX7000显微镜进行切片和图像采集。

\subsection{Deep Learning in Tissue Sectioning}

The application of deep learning technologies in the biomedical field has achieved significant advancements. Deep learning models excel in tasks such as image classification, object detection, and segmentation, providing powerful tools for research and diagnostics in biomedical laboratories.

Lorena Guachi-Guachi proposed a method utilizing CNN networks to identify and refine tissue sections. This approach represents an innovative application of deep learning that can enhance the precision of tissue preparation and analysis \cite{LR.7}.

In the book \textit{Biomedical Texture Analysis}, Vincent Andrearczyk introduced a CNN architecture specifically designed for texture analysis, which significantly improves the accuracy of classifying biological tissues compared to traditional architectures \cite{LR.8}. This development demonstrates the potential of deep learning to enhance the detailed analysis of tissue characteristics, which is crucial for accurate diagnostics and research.

Yan Xu suggested that features extracted from CNNs trained on the large natural image database, ImageNet, can be transferred to histopathological images of tissues. This provides a viable approach for implementing transfer learning, which can greatly enhance the efficiency of tissue image classification and analysis \cite{LR.9}.

Based on the literature, deep learning technology holds broad prospects for application in image classification and analysis of tissue sections. By leveraging deep learning models, efficient identification and classification of tissue samples can be achieved, providing strong support for optimizing sectioning parameters.

This section underscores the transformative impact of deep learning on the field of tissue sectioning, promising significant improvements in the accuracy and utility of histological analyses.

% 在生物医学领域,深度学习技术的应用已取得了显著的进步。深度学习模型在图像分类、对象检测和分割等任务中表现出色,为生物医学实验室的研究和诊断提供了强大的工具。

% Lorena Guachi-Guachi 提出了一种利用 CNN 网络识别和精炼组织切片的方法。这种方法代表了深度学习的创新应用,可以提高组织准备和分析的精度 \cite{LR.7}。

% 在《生物医学纹理分析》一书中,Vincent Andrearczyk 介绍了一种专为纹理分析设计的 CNN 架构,与传统架构相比,这种架构显著提高了生物组织分类的准确性 \cite{LR.8}。这一发展展示了深度学习提高组织特性详细分析的潜力,这对于准确的诊断和研究至关重要。

% Yan Xu 提出,从在大型自然图像数据库 ImageNet 上训练的 CNN 中提取的特征可以转移到组织的病理学图像上。这为实施转移学习提供了一种可行的方法,可以大大提高组织图像分类和分析的效率 \cite{LR.9}。

% 根据文献,深度学习技术在组织切片的图像分类和分析中有广阔的应用前景。通过利用深度学习模型,可以实现组织样本的有效识别和分类,为优化切片参数提供了强大的支持。

% 这一部分强调了深度学习对组织切片领域的变革性影响,预示着在组织学分析的准确性和实用性方面的显著改进。




% \subsection{关于切片组织的深度学习}

% 深度学习技术在生物医学领域的应用已经取得了显著进展。深度学习模型在图像分类、目标检测和分割等任务中表现出色,为生物医学实验室的研究和诊断提供了强大的工具。

% Lorena Guachi-Guachi 提出了一种使用cnn网络对组织切片进行识别并进行修整。\cite{LR.7}

% % 应用于切片术的卷积神经网络:识别石蜡包埋组织块的修剪末端切割程序
% % https://www.sciencedirect.com/science/article/abs/pii/S0952197623011478


% 在Biomedical Texture Analysis一书中,Vincent Andrearczyk提出了一种专门用于纹理分析的cnn架构,相比其它传统架构能够显著提高生物组织的分类准确性.\cite{LR.8}

% % 第 4 章-纹理分析中的深度学习及其在组织图像分类中的应用
% % https://www.sciencedirect.com/science/article/abs/pii/B9780128121337000041


% Yan Xu提出,从大型自然图像数据库 ImageNet 训练的 CNN 中提取的特征能够转移到组织病理学图像中,这为我们实现迁移学习提供了一种可行的思路。\cite{LR.9}

% % 通过深度卷积激活特征进行大规模组织病理学图像分类、分割和可视化
% % https://link.springer.com/article/10.1186/s12859-017-1685-x

% 根据以上文献,深度学习技术在组织切片的图像分类和分析中具有广泛的应用前景。通过利用深度学习模型,可以实现对组织样本的高效识别和分类,为优化切片参数提供有力支持。

% \FloatBarrier % Now figures cannot float above section title
