\pagenumbering{arabic}
\section{Introduction and background}
\label{sec:introduction}

\subsection{Background}

As the fundamental units of life, human research into cells and tissues has never ceased. Biological tissue sections, serving as crucial means for the direct observation of cellular morphology and structure, are essential for biomedical research and clinical diagnosis. A complete and usable tissue section is of great importance to researchers and physicians, as it provides vital information about cell structure, tissue morphology, and pathological changes. Within this, the quality of the section is of paramount importance.

Traditional manual sectioning methods are time-consuming and prone to variability, hence the emergence of automatic microtomes has provided a solution to these issues. For different biological tissues, varying cutting parameters can yield differing results, both positive and negative. Therefore, to enhance the utilization rate of biological sections and increase the production of high-quality specimens, determining optimal cutting parameters for specific tissues remains a goal.

Machine learning and deep learning have achieved significant success in the fields of computer vision and image processing. Machine learning is defined as a series of methods that can automatically detect patterns in data, which are then used to predict future outcomes or make decisions \cite{1.1}. In this paper, we integrate advanced image analysis and machine learning techniques to identify section quality and then evaluate the quality of tissue samples under different sectioning parameters.

\subsection{Introduction}

\textbf{Project Overview}

This project aims to optimize the cutting parameters of biological tissue microtomes, which are crucial devices in biomedical research and clinical diagnostics. The objective is to enhance the precision and efficiency of tissue sample preparation by determining the optimal slicing conditions. By collecting tissue samples under various cutting parameters and conducting subsequent manual image classification, this study employs deep learning techniques to analyze and predict the most effective cutting parameters. This work not only promises to improve the quality of tissue samples for microscopic examination but also helps to simplify laboratory workflows, thereby advancing biological and medical sciences.

\textbf{Objectives:}

\begin{enumerate}
    \item Collect a comprehensive dataset of tissue samples sliced under different parameters.
    \item Employ artificial image classification to categorize the quality and characteristics of these samples.
    \item Develop and train a deep learning model capable of assessing tissue sample quality.
    \item Use the model's insights to determine the optimal cutting parameters for the tissue slicer.
    \item Validate the model's predictions through empirical testing and refinement.
\end{enumerate}


\subsection{Structure of the Report}

The project is structured into several chapters, each aimed at systematically exploring research background, methodologies, experimental work, results presentation, discussions and conclusions, as well as considerations for project management, sustainability, and health safety:

\textbf{Introduction and Background} - This chapter outlines the project's goals, objectives, and structural arrangements. It briefly introduces the motivation and necessity for the research, as well as the adopted technical protocols and standards.

\textbf{Literature Review} - Provides an in-depth discussion on biological tissue sections, image classification, and the application of deep learning in the preparation of biological samples. This section positions the current study within the context of existing research.

\textbf{Methods and Theory} - Detailed description of experimental methods, theoretical frameworks, and specific plans for data collection and processing.

\textbf{Experimental Work/Analysis Investigation/Design} - Describes in detail the experimental design, implementation, and analytical survey. It elaborates on the strategies and methods adopted to achieve the project's objectives.

\textbf{Presentation of Experimental or Analysis Results/Final Constructed Product Description} - This chapter presents experimental data, analysis results, or final design products, detailing the outcomes of experiments or designs.

\textbf{Discussion and Conclusion} - Analyzes the results, discussing their scientific significance and practical value. This chapter also provides conclusions from the research and suggests potential directions for future studies.

\textbf{Project Management, Sustainability, and Health Safety Considerations} - Discusses project management strategies, sustainability issues, and health safety measures to ensure efficient and safe conduct of the research work.

\textbf{References} - Lists all the referenced literature, supporting the research and providing a foundation for the study.



\textbf{Assumptions and Technical Specifications}

The project is based on several key assumptions and technical protocols, which include:

\begin{enumerate}
    \item Consistency in the properties of tissue samples across different batches.
    \item The reliability and accuracy of the biological tissue microtome and imaging equipment.
    \item The adequacy of deep learning models in interpreting complex biological image data.
\end{enumerate}

Technical specifications regarding tissue microtome settings, image classification standards, and deep learning architecture are detailed in the Methods and Theory section.
