\pagenumbering{arabic}
\section{Introduction and background}
\label{sec:introduction}
\subsection{Background}

As the fundamental units of life, human research into cells and tissues has never ceased. Biological tissue sections, serving as crucial means for the direct observation of cellular morphology and structure, are essential for biomedical research and clinical diagnosis. A complete and usable tissue section is of great importance to researchers and physicians, as it provides vital information about cell structure, tissue morphology, and pathological changes. Within this, the quality of the section is of paramount importance.

Traditional manual sectioning methods are time-consuming and prone to variability, hence the emergence of automatic microtomes has provided a solution to these issues. For different biological tissues, varying cutting parameters can yield differing results, both positive and negative. Therefore, to enhance the utilization rate of biological sections and increase the production of high-quality specimens, determining optimal cutting parameters for specific tissues remains a goal.

Machine learning and deep learning have achieved significant success in the fields of computer vision and image processing. Machine learning is defined as a series of methods that can automatically detect patterns in data, which are then used to predict future outcomes or make decisions \cite{1.1}. In this paper, we integrate advanced image analysis and machine learning techniques to identify section quality and then evaluate the quality of tissue samples under different sectioning parameters.

\subsection{Introduction}
\textbf{Project Overview}

This project aims to optimize the cutting parameters of biological tissue microtomes, which are crucial devices in biomedical research and clinical diagnostics. The objective is to enhance the precision and efficiency of tissue sample preparation by determining the optimal slicing conditions. By collecting tissue samples under various cutting parameters and conducting subsequent manual image classification, this study employs deep learning techniques to analyze and predict the most effective cutting parameters. This work not only promises to improve the quality of tissue samples for microscopic examination but also helps to simplify laboratory workflows, thereby advancing biological and medical sciences.

\textbf{Objectives:}

\begin{enumerate}
    \item Collect a comprehensive dataset of tissue samples sliced under different parameters.
    \item Employ artificial image classification to categorize the quality and characteristics of these samples.
    \item Develop and train a deep learning model capable of assessing tissue sample quality.
    \item Use the model's insights to determine the optimal cutting parameters for the tissue slicer.
    \item Validate the model's predictions through empirical testing and refinement.
\end{enumerate}


\subsection{Structure of the Report}
The report is organized into multiple chapters, each focusing on specific aspects of the research on optimizing biopsy parameters using deep learning:

\textbf{Introduction and Background} - This initial chapter outlines the project's goals and framework, provides the motivation for the research, and describes the technical protocols and standards employed.

\textbf{Literature Review} - An extensive review of the relevant literature on biological tissue sections, image classification, and deep learning applications in biological sample preparation. This section contextualizes the current study within existing research.

\textbf{Methods and Theory} - A comprehensive description of the experimental methods, theoretical frameworks, and plans for data collection and processing.

\textbf{Experimental Work/Analysis Investigation/Design} - Details the experimental design, implementation, and analytical survey, explaining the strategies and methods used to meet the project objectives.

\textbf{Presentation of Experimental or Analysis Results/Final Constructed Product Description} - This chapter documents the experimental data, analysis results, or descriptions of the final design products, elaborating on the outcomes.

\textbf{Discussion and Conclusion} - Discusses the results in terms of their scientific significance and practical implications, draws conclusions, and suggests future research directions.

\textbf{Project Management, Sustainability, and Health Safety Considerations} - Covers project management strategies, addresses sustainability and health safety issues to ensure the research is conducted efficiently and safely.

\textbf{References} - Compiles all literature referenced throughout the research, supporting the study's foundation.


\textbf{Assumptions and Technical Specifications} - The project relies on several assumptions:

\begin{enumerate}
    \item Uniform properties of tissue samples across different batches.
    \item Reliability and accuracy of the biological tissue microtome and imaging equipment.
    \item Effectiveness of deep learning models in interpreting complex biological image data.
\end{enumerate}

Technical details regarding tissue microtome settings, image classification standards, and deep learning architecture are thoroughly described in the Methods and Theory section.
