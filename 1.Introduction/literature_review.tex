\section{Literature review}


这篇文献综述探讨了生物组织切片中技术的融合,特别关注图像分类和深度学习在优化切片参数方面的应用。它旨在突出重要的进展,确定当前方法学中存在的差距,并为拟议的项目奠定基础。

\subsection{切片机与显微镜}

近年来,自动切片机的出现显著简化了切片过程,并提高了切片的质量。

Zimmermann在文章"Improved reproducibility in preparing precision-cut liver tissue slices"中,主张使用新的Leica振动刀来提高大鼠、小鼠和人体组织切片的精度和重复性 \cite{LR.1}。

在这个实验中,我们使用Epredia提供的HM355S切片机进行切片。这台机器是生物组织切片研究的流行设备,许多实验和论文都使用了这台设备进行切片。

Elzbieta Klimuszko使用HM355S切片机切割牙齿,以研究牙釉质中的钙和镁含量 \cite{LR.2}。

Andelko Hrzenjak也使用HM355S切片机切割病理性子宫内膜组织,以研究子宫内膜癌发展的机制 \cite{LR.3}。

同样,显微镜的选择也至关重要。在这个实验中,我们使用Keyence的VHX7000显微镜进行图像采集。它能够捕获生物组织切片的图像(例如,小鼠前列腺细胞 \cite{LR.4}),以及无机材料(如陶瓷 \cite{LR.5},玻璃 \cite{LR.6})。

实验将使用HM355s切片机和VHX7000显微镜进行切片和图像采集。这种设置确保了设备选择和技术应用的最佳配合,以提高组织切片过程的精度和效率,支持研究项目的总体目标。

% \subsection{深度学习的原理}










\subsection{关于切片组织的深度学习}

在生物医学领域,深度学习技术的应用已取得了显著的进步。深度学习模型在图像分类、对象检测和分割等任务中表现出色,为生物医学实验室的研究和诊断提供了强大的工具。

Lorena Guachi-Guachi 提出了一种利用 CNN 网络识别和精炼组织切片的方法。这种方法代表了深度学习的创新应用,可以提高组织准备和分析的精度 \cite{LR.7}。

在《生物医学纹理分析》一书中,Vincent Andrearczyk 介绍了一种专为纹理分析设计的 CNN 架构,与传统架构相比,这种架构显著提高了生物组织分类的准确性 \cite{LR.8}。这一发展展示了深度学习提高组织特性详细分析的潜力,这对于准确的诊断和研究至关重要。

Yan Xu 提出,从在大型自然图像数据库 ImageNet 上训练的 CNN 中提取的特征可以转移到组织的病理学图像上 \cite{LR.9}。这为实施转移学习提供了一种可行的方法,可以大大提高组织图像分类和分析的效率。

根据文献,深度学习技术在组织切片的图像分类和分析中有广阔的应用前景。通过利用深度学习模型,可以实现组织样本的有效识别和分类,为优化切片参数提供了强大的支持。

这一部分强调了深度学习对组织切片领域的变革性影响,预示着在组织学分析的准确性和实用性方面的显著改进。

\FloatBarrier % Now figures cannot float above section title
