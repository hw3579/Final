\section{Presentation of experimental or analytical results/descriptions of final constructed product}
在这一节我们主要讨论模型的测试结果和模型进一步改进的空间。


\subsection{带入测试集验证准确度}

在这里我们将训练好的模型应用到额外准备好的测试集上,计算模型的准确度。

\autoref{tab:model_accuracy}是模型在测试集上的准确度:(准确度定义为标签与模型预测一致的样本数占总样本数的比例)

\begin{figure}[htbp]
    \centering
    \begin{minipage}{0.45\textwidth}
        \centering
        \captionof{table}{Model accuracy on test set}
        \begin{tabular}{cc}
            \toprule
            Category & Accuracy(\%) \\
            \midrule
            Normal & 98.4 \\
            Horizontal Line & 95.6 \\
            Vertical Line & 80.0 \\
            Slope & 96.1 \\
            Other & 95.2 \\
            \bottomrule
        \end{tabular}
        \label{tab:model_accuracy}
    \end{minipage}
    \begin{minipage}{0.45\textwidth}
        \centering
        \includegraphics[width=\textwidth]{./fig/assistplot/accuracy.eps}
        \captionof{figure}{Model Accuracy on Test Set}
        \label{fig:accuracy_histogram}
    \end{minipage}
\end{figure}


可见,模型预测结果在normal上表现较好,而在vertical\_line上表现较差。这可能是由于用于测试的样本量较少,导致模型学习不足,难以判断。


\subsection{模型提高(改变输入分辨率)}

在这里我们讨论模型的进一步提高的空间。

将高分辨率图片缩放为InceptionV3模型默认的299x299大小的确可能导致信息和细节的丢失,特别是对于原始分辨率远高于此标准的图像。例如,从VHX7000设备采集的2880x2160分辨率图像就含有大量的细节,直接缩放可能不利于模型捕捉到所有的细微差别,尤其是在医学影像或其他细节丰富的领域。

改变模型的输入层接受更大的图片尺寸是一个潜在的解决方案。这样做的优势是它允许模型处理更高分辨率的图像,保留更多的原始信息和细节,可能导致更好的性能和更高的准确度。此外InceptionV3的架构设计有助于处理更大图片,因为其含有多个大小不一的卷积核,这使得它能够捕捉不同尺度的特征。

受制于实验室机器性能(显存为16G),在这里将图像缩放到分别为原图像的0.4倍,即1152*864,进行再一次训练。

新的模型为model4,训练效果如下所示:

\begin{figure}[H]
    \centering
    \begin{minipage}{0.45\textwidth}
        \centering
        \includegraphics[width=\textwidth]{./fig/model4/accuracy4.eps}
        \caption{Model-4 accuracy}
        \label{fig:model4_accuracy}
    \end{minipage}
    \begin{minipage}{0.45\textwidth}
        \centering
        \includegraphics[width=\textwidth]{./fig/model4/loss4.eps}
        \caption{Model-4 loss}
        \label{fig:model4_loss}
    \end{minipage}
\end{figure}

观察训练准确度和损失随步长的变化可以发现,模型的性能有惊人的明显提升。训练和验证准确度都接近1,同时验证损失降至15\%左右,这通常表明模型具有很强的泛化能力。这种情况下,模型不仅在训练数据上表现出色,而且能够很好地泛化到新的、未见过的数据上。

将其再一次带入测试集进行准确度评估,结果如\autoref{tab:model_accuracy2}:

\begin{table}
    \centering
    \caption{model accuracy on test set}
    \begin{tabular}{cccccc}
        \toprule
        & normal & horizental\_line & vertical\_line & slope & other \\
        \midrule
        accuracy(\%) & 98.4 & 96.7 & 85.6 & 96.5 & 96.5 \\
        \bottomrule
    \end{tabular}
    \label{tab:model_accuracy2}
    \end{table}

对比修改分辨率前后的模型准确度,可以发现模型的准确度虽有提升但并不显著的,可能是由于准确度已经很接近于1,提升的空间较小导致的。


\subsection{探究机器的最佳切削角}

使用预先准备好的各个切削角度的图片,从8-12度,每0.5度一个样本,一共9组数据,每组数据包含100张图片,使用模型4来评估每组的良品率。此时,找到良品率最高的数据组,即为该机器的最佳切削角度。

\begin{figure}[htbp]
    \centering
    \begin{minipage}{0.4\textwidth}
        \centering
        \captionof{table}{Normal accuracy on different angle}
        \begin{tabular}{cc}
            \toprule
            Angle & Accuracy(\%) \\
            \midrule
            8 & 80 \\
            8.5 & 81.5 \\
            9 & 83.5 \\
            9.5 & 93.3 \\
            10 & 96.6 \\
            10.5 & 88.8 \\
            11 & 84.2 \\
            11.5 & 66.6 \\
            12 & 62.2 \\
            \bottomrule
        \end{tabular}
        \label{tab:model_accuracy_angle}
    \end{minipage}
    \begin{minipage}{0.55\textwidth}
        \centering
        \includegraphics[width=\textwidth]{./fig/assistplot/angle_accuracy.eps}
        \captionof{figure}{Model Accuracy on Different Angle}
        \label{fig:angle_accuracy_histogram}
    \end{minipage}
\end{figure}

从\autoref{tab:model_accuracy_angle}中呈现的数据可以看出,为了获得最高质量组织切片的最高产量,最佳切割角度是10度,这显示出了令人印象深刻的96.6%的准确率。

此外,如\autoref{fig:angle_accuracy_histogram}所示,为了保持至少80%的切片质量率,切割角度应设定在9度到10.5度之间。这个范围不仅确保了高质量切割的比率,而且还提供了一些机器设置的灵活性,以适应可能的组织类型或状况的变化。

\subsection{模型通用性}

到目前为止,我们的实验使用的是鱼的卵巢组织切片。在实际应用中,我们可能会遇到各种各样的组织样本,包括其他器官或来自不同动物的标本。因此,评估我们的模型在各种组织类型中的泛化能力是至关重要的。

我们已经准备了一个新的数据集,包括鱼的肺组织切片,分为四类:好、正常、坏和其他。这些类别在下面的图中有所展示: (从\autoref{fig:good_fish_lung}到\autoref{fig:other_fish_lung})

\begin{figure}[H]
    \centering
    \begin{minipage}{0.24\textwidth}
        \centering
        \includegraphics[width=\textwidth]{./fig/fish_lung/good20240313_144138.jpg}
        \caption{Good fish lung}
        \label{fig:good_fish_lung}
    \end{minipage}
    \begin{minipage}{0.24\textwidth}
        \centering
        \includegraphics[width=\textwidth]{./fig/fish_lung/normal20240313_141726.jpg}
        \caption{Normal fish lung}
        \label{fig:noraml_fish_lung}
    \end{minipage}
    \begin{minipage}{0.24\textwidth}
        \centering
        \includegraphics[width=\textwidth]{./fig/fish_lung/bad20240313_140952.jpg}
        \caption{Bad fish lung}
        \label{fig:bad_fish_lung}
    \end{minipage}
    \begin{minipage}{0.24\textwidth}
        \centering
        \includegraphics[width=\textwidth]{./fig/fish_lung/other20240313_141858.jpg}
        \caption{Other fish lung}
        \label{fig:other_fish_lung}
    \end{minipage}
\end{figure}

训练的准确度和损失如\autoref{fig:model5_acc}和\autoref{fig:model5_loss}所示。
\begin{figure}[H]
    \centering
    \begin{minipage}{0.45\textwidth}
        \centering
        \includegraphics[width=\textwidth]{./fig/fish_lung/accuracy5.eps}
        \caption{Model-5 accuracy}
        \label{fig:model5_acc}
    \end{minipage}
    \begin{minipage}{0.45\textwidth}
        \centering
        \includegraphics[width=\textwidth]{./fig/fish_lung/loss5.eps}
        \caption{Model-5 loss}
        \label{fig:model5_loss}
    \end{minipage}
\end{figure}

通过观察图片可以得出:
模型5的训练和验证准确度迅速上升并保持在高位,表明模型在这两个数据集上均有良好表现,损失图显示训练损失快速降低并趋于零,而验证损失在初始阶段出现尖峰后迅速降低并稳定,整体来看,这些迹象表明模型具有较好的拟合能力和泛化性能。

将其带入测试集进行测试,结果如\autoref{fig:accuracy_histogram2}所示:

\begin{figure}[H]
    \begin{minipage}{0.45\textwidth}
        \centering
        \captionof{table}{Model accuracy on test set}
        \begin{tabular}{ccccc}
            \toprule
            label & accuracy(\%) \\
            \midrule
            bad & 94.1 \\
            good & 98.2 \\
            normal & 94.7 \\
            other & 95.0 \\
            \bottomrule
        \end{tabular}
        \label{tab:model_accuracy3}
    \end{minipage}
    \begin{minipage}{0.45\textwidth}
        \centering
        \includegraphics[width=\textwidth]{./fig/assistplot/angle_accuracy2.eps}
        \caption{Model Accuracy on Test Set}
        \label{fig:accuracy_histogram2}
    \end{minipage}
\end{figure}

该模型在所有标签上的准确率超过90\%,显示出其强大的性能和显著的泛化能力。这表明该模型可以有效地分类不同类型的组织切片,可能使其成为各种生物医学成像应用的多功能工具。该模型在不同组织类型中的稳健性强调了其在组织质量评估和分类任务中的潜力.



\FloatBarrier