\section{Discussion and conclusions}
\label{sec:results}

\subsection{Discussion of results}


正如上文中所详述,这里将详细讨论研究过程中的发现和结果。

在第4节中,我们试图建立模型,尝试了从简单的cnn模型开始,之后进行图像的预处理,发现效果都不理想的情况下尝试了迁移学习的方法,得出采用IncetionV3模型的迁移学习效果最好。


在这其中值得注意的是伴随着不同模型的尝试,当模型参数调整或模型架构变得更加复杂时(例如 InceptionV3),模型性能显着提高,即验证集的准确度越来越高,损失越来越低。

此外,对比模型系列1和2,我们发现在在图像分类的任务中使用预处理图像的手段去辅助机器提取特征的行为不是非常有效。采用图像处理有可能会导致重要细节及信息的丢失,进而影响机器对特征的提取,从而影响模型的准确度和性能。

在第5节中,我们对模型进行了应用测试,首先选取了额外的测试集测试模型的准确度,发现模型在所有测试集上的准确度均大于85\%。之后带入使用模型对不同切削角进行评估,发现若要在保证切削质量为百分之 80 的情况下,切削角度应在 9 度到 10.5 度之间。最后,我们使用了另外的鱼的肺泡切片的图像数据集对模型进行二次验证,发现模型对于测试集的标签预测准确率均在90\%以上,反应模型能够很好地应用于其他数据集。

最终的采取的模型如下:
\begin{center}

    \begin{tikzpicture}[node distance=1.5cm,
        box/.style={
            rectangle,
            rounded corners,
            draw=black, very thick,
            text width=15em,
            minimum height=2em,
            text centered},
        arrow/.style={
            thick,
            ->,
            >=stealth}
        ]
    
        \node (collect) [box] {输入层:1152*864};
        \node (mix) [box, below of=collect] {基础模型:InceptionV3};
        \node (train) [box, below of=mix] {全局平均池化层};
        \node (test) [box, below of=train] {全连接层(节点个数取决于标签)-输出层};
        \node (evaluate) [box, below of=test] {学习率:1e-4,优化器:Adam};
        \node (rate) [box, below of=evaluate] {损失函数:交叉熵,评估指标:准确率};
        \node (improve) [box, below of=rate] {早停:启用};
        
        \draw [arrow] (collect) -- (mix);
        \draw [arrow] (mix) -- (train);
        \draw [arrow] (train) -- (test);
        \draw [arrow] (test) -- (evaluate);
        \draw [arrow] (evaluate) -- (rate);
        \draw [arrow] (rate) -- (improve);
        
        \end{tikzpicture}
\end{center}

\subsection{Future work}

\textbf{分类方法的改进}

本研究提供了多个改进和深化的路径。尤其是,所采用的分类方法已经展现出理想的结果;然而,不足的是,五个分类仍旧是不充足的。通过提升分类的多样性,可以进一步得到切削角度和样本良品率的关系,进而提供更精确的分析。

此外,在拥有足够多的分类点的情况下,可以考虑从分类框架向线性分析方法过渡。在分类种类足够多的情况下,可以近似认为这是线性关系下的离散点-即可以通过线性拟合来得到最佳切削角度和样本良品率的线性关系。

在统计和数据科学中,当分类问题能够被线性化(及分类的标签能够被离散化,如上文中的切削角度)的时候-通常将其称之为线性判别分析(Linear discriminant analysis)。

在模型显示出显著能力和鲁棒性的情况下,采用线性判别分析——特别是在确定最优切割角度方面——可能呈现出一种更精细的方法,用于关联组织质量与切割参数。这样的方法有可能简化切割参数的预测准确性,并有助于更精确地控制组织切割过程。

当然,将分类问题转变为线性判别分析问题的难度是非常大的。其中一个关键的问题是,二分类问题的模型预测输出结果是概率值,而线性回归问题的本质是找到切削角度和切片质量的关系,即自变量是切削角度,因变量是切削质量。此外,显然切削角度和切片质量也不是简单的一元线性关系,模型需要处理的数据将会非常复杂。因此,这一方面的改进需要更多的理论和实践的探索。

此外,线性模型的训练和验证需要巨量的数据(甚至高出几个数量级),而这些数据采集和收集将会是一个非常漫长且困难的过程。不仅如此,线性回归模型的训练和验证也需要更多的计算资源。目前在采用tensorflow框架下选择InceptionV3模型,输入分辨率为1152*864的情况下,显卡的显存已经达到了极限,因此在这一方面的改进需要更多的硬件和更加强大的计算资源。所以,这一方面的改进是一个长期的目标,需要更多的资源和时间。

关于线性判别分析的研究,则是一个更加深入的方向,需要更多的理论和实践的探索。
比如Jie Wen 提到的Robust Sparse Linear Discriminant Analysis\cite{6.1}. 该方法在线性判别分析的基础上,引入了稀疏性,使得模型更加稳健。


% (如果可以增加相关论文证明观点:二分类和线性回归问题的对比和性能开销)


\textbf{性能提升与优化}

随着本研究目标向大规模应用前进,性能提升成为了必须面临的重要挑战。性能优化不仅仅局限于算法效率的提升,还包括模型框架的扩展性、稳定性和部署能力的增强以及底层的语言和代码优化。

显然,针对计算资源的利用和调度策略,我们应当寻求更高效的计算框架和并行处理算法。例如,利用分布式计算资源,可以显著缩短模型训练的时间,并提高处理大型数据集时的效率。同时,考虑到能耗和计算成本的约束,优化模型的计算结构和参数设置,以期达到在有限资源下最大化计算输出的目的。

《Analysis of the Application Efficiency of TensorFlow and PyTorch in Convolutional Neural Network》这篇文章总结了两种主流框架tensorflow和pytorch在卷积神经网络中的区别,其中tensorflow具有更低的误差率和更小的收敛步长,而pytorch具有更快的训练速度\cite{6.2}。


% https://www.mdpi.com/1424-8220/22/22/8872


Pascal Fua在《Comparing Python, Go, and C++ on the N-Queens Problem》一文中给出了一种优化深度学习性能的方法。通过对比解释性语言Python、编译型语言go和C++在N皇后问题上的性能,发现runtime语言在处理循环和数据流时具有明显优势,因此在深度学习中,一个提升性能的方法就是使用如numba编译器与cython,pybind进行编译\cite{6.3}。

% https://arxiv.org/abs/2001.02491

% (论文:tf和torch的对比,libtf libtf,opencv c++和py的对比)

\textbf{切片流程的优化}

在我们的研究中,我们还发现,如果能在切片过程中实时评估切削质量,并根据评估结果调整切削参数,将会在流程上显著提升切片质量。

具体的反馈调节流程包括,通过在切片机器上方安装摄像头对从刀片上切削下来的样本进行数据采集,然后将照片输入模型进行实时评估,然后根据评估结果控制机器的切削的给进速度和角度等参数,改进下一个切片样本的质量,实现样本质量的可控性和保证样本质量。

显然,如果要达成这一切仍然是个挑战。首先,需要一个清晰的相机和高效的实时图像处理系统,能够对图像数据进行采集;然后需要一个预训练好的模型和具有强大性能的电脑,能够快速内对图像进行评估并且给出结果,并根据结果确定需要给机器提供修改的参数;最后需要一个高效的控制接口,能够保证修改后的参数能够及时传递给切片机器。最后,这整个系统对时间的要求也是非常高的,需要在两次切削之间内完成所有的操作。

一个很好的例子就是《Convolutional neural networks applied to microtomy: Identifying the trimming-end cutting routine on paraffin-embedded tissue blocks》。在这篇文章当中,作者提出了一种使用cnn网络对组织切片进行识别并进行修整的方法,通过在切片机器上安装摄像头,对切片过程进行实时监控,然后将图像输入cnn网络进行识别,最后根据识别结果调整切片机器的参数,实现了切片过程的自动化\cite{6.4}。通过将切片机,相机,深度学习模型等独立部分联系在一起,为我们在切削过程中实时评估切削质量和修改切削参数提供一个可行的解决方案。


\subsection{Conclusions}

这项调查研究使我们对生物医学组织切片机的活检参数优化有了重要的洞见。我们对各种切割角度的专注实验,结合深度学习技术的严格应用,不仅提高了我们对组织采样的理解,而且引发了未来如何处理此类任务的范式转变。

从全面的实验工作和分析调查中可以明显看出,通过对复杂的卷积神经网络的迁移学习,产生了一个能够以高精度评估组织样本质量的强大框架。通过将模型应用于各种组织类型,进一步证实了其适应性,这突显了其在组织病理学领域作为一种多功能工具的广泛适用性和潜力。

此外,在寻找最佳模型的过程中,还探究了图像属性和模型复杂性之间的关系。我们发现,将图像进行一定程度的预处理后作为输入层可能会无意中丢弃重要的细节和信息,这说明图像预处理是可以被省略的。此外,还发现通过调整输入层为更高分辨率的图像,模型的准确性和预测性能显著提高,这说明保持原始数据的完整性的重要性。


总的来说,这个项目的结果不仅强化了深度学习在生物医学器械应用中的不可或缺性,而且为继续探索和创新优化组织切片技术奠定了基础,最终有助于推进生物切片技术的改良。








\FloatBarrier % Now the table doesn't flow over to any other sections