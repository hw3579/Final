\section{Discussion and conclusions}
\label{sec:results}

\subsection{讨论}


正如上文中所详述,这里将详细讨论研究过程中的发现和结果。

在第4节中,我们试图建立模型,尝试了从简单的cnn模型开始,之后进行图像的预处理,发现效果都不理想的情况下尝试了迁移学习的方法,得出采用IncetionV3模型的迁移学习效果最好。


在这其中值得注意的是伴随着不同模型的尝试,当模型参数调整或模型架构变得更加复杂时(例如 InceptionV3),模型性能显着提高,即验证集的准确度越来越高,损失越来越低。

此外,对比模型系列1和2,我们发现在在图像分类的任务中使用预处理图像的手段去辅助机器提取特征的行为不是非常有效。采用图像处理有可能会导致重要细节及信息的丢失,进而影响机器对特征的提取,从而影响模型的准确度和性能。

在第5节中,我们对模型进行了应用测试,首先选取了额外的测试集测试模型的准确度,发现模型在所有测试集上的准确度均大于85\%。之后带入使用模型对不同切削角进行评估,发现若要在保证切削质量为百分之 80 的情况下,切削角度应在 9 度到 10.5 度之间。最后,我们使用了另外的鱼的肺泡切片的图像数据集对模型进行二次验证,发现模型能够很好地应用于其他数据集。(贴数据)


\subsection{Future work}

\textbf{分类方法的改进}

本研究提供了多个改进和深化的路径。尤其是,所采用的分类方法已经展现出理想的结果;然而,不足的是,五个分类仍旧是不充足的。通过提升分类的多样性,可以进一步得到切削角度和样本完整度的关系,进而提供更精确的分析。

此外,在拥有足够多的分类点的情况下,可以考虑从二元分类框架向线性分析方法过渡。在分类种类足够多的情况下,可以近似认为这是线性关系下的离散点-即可以通过线性拟合来得到最佳切削角度和样本完整度的线性关系。

在模型显示出显著能力和鲁棒性的情况下,采用线性分析——特别是在确定最优切割角度方面——可能呈现出一种更精细的方法,用于关联组织质量与切割参数。这样的方法有可能简化切割参数的预测准确性,并有助于更精确地控制组织切割过程。

当然,将分类问题转变为线性回归问题的难度是非常大的。其中一个关键的问题是,线性回归模型的训练和验证需要巨量的数据(甚至高出几个数量级),而这些数据采集和收集将会是一个非常漫长且困难的过程。不仅如此,线性回归模型的训练和验证也需要更多的计算资源,目前在采用tensorflow框架下选择InceptionV3模型,输入分辨率为1152*864的情况下,显卡的显存已经达到了极限,因此在这一方面的改进需要更多的硬件和更加强大的计算资源。所以,这一方面的改进是一个长期的目标,需要更多的资源和时间。(如果可以增加相关论文证明观点:二分类和线性回归问题的对比和性能开销)


\subsection{性能提升与优化}

随着本研究目标向大规模应用前进,性能提升成为了必须面临的重要挑战。性能优化不仅仅局限于算法效率的提升,还包括模型框架的扩展性、稳定性和部署能力的增强以及底层的语言和代码优化。

显然,针对计算资源的利用和调度策略,我们应当寻求更高效的计算框架和并行处理算法。例如,利用分布式计算资源,可以显著缩短模型训练的时间,并提高处理大型数据集时的效率。同时,考虑到能耗和计算成本的约束,优化模型的计算结构和参数设置,以期达到在有限资源下最大化计算输出的目的。(论文:tf和torch的对比,libtf libtf,opencv c++和py的对比)



\subsection{总结}











\FloatBarrier % Now the table doesn't flow over to any other sections