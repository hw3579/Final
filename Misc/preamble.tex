%'''
\usepackage{amsmath}   % Package for math files
\usepackage{txfonts}
\usepackage{graphicx}  % Place figures
\usepackage{caption}   % Place captions in tables and figures
\usepackage{subcaption}% 
\usepackage{placeins}  % \FloatBarrier so figures can't float beyond some point in text
\usepackage{fullpage}  % Uses more of the page
\usepackage{float}     % Able to make figures and tables float
%表格自动生成
\usepackage{subfloat}
\usepackage{subcaption} 
\usepackage{subfiles}  
\usepackage{hyperref} 
 % Ability to click on references like equations, figures, sections etc. \ref{eq:my_eq} clickable
\usepackage{placeins}  
% \FloatBarrier so figures can't float beyond some point in text
\usepackage[T1]{fontenc} 
%英文字体定义
\usepackage{setspace}
\usepackage{fontspec}
\usepackage{verbatim}

\usepackage{tikz}
\bibliographystyle{unsrt}
%''' 
%以上为宏包的引用


%'''
\renewcommand {\thetable} {\thechapter{}.\arabic{table}}
%表格根据章节自动命名
\renewcommand {\thefigure} {\thechapter{}.\arabic{figure}}
%图片根据章节自动命名
\numberwithin{figure}{section}
\numberwithin{table}{section}
\numberwithin{equation}{section}
%以上三个皆控制重命名
\newcommand{\figref}[1]{\figurename~\ref{#1}} 
%Nice reference to figures

\usepackage[framemethod=TikZ]{mdframed}
\usepackage{url}   % 网页链接
\usepackage{subcaption} % 子标题
\renewcommand{\contentsname}{Contents}
\renewcommand{\listfigurename}{List of Figures}
\renewcommand{\listtablename}{List of Tables}
\renewcommand{\indexname}{Index}
\renewcommand{\figurename}{Figure}
\renewcommand{\tablename}{Table}
\renewcommand{\refname}{References}
\usepackage{tocloft}
\usepackage{fancyhdr} 
%'''
%以上为renewcommand的使用
%页面格式编辑
\geometry{top=2.54cm, bottom=2.54cm, left=2.54cm, right=2.54cm, headheight=15pt}
%%以下部分编辑页眉与页脚
%\pagestyle{fancy}
%\fancyhf{} % 清除所有的页眉和页脚
%\setlength{\headheight}{15pt} % 增加页眉的高度
%\setlength{\headsep}{25pt}    % 增加页眉和正文之间的距离
%\fancyhead[R]{Student Name: Zhan Ruixin}
%\fancyhead[L]{Preliminary Report (ECM3175)} % 页眉左侧
%\fancyfoot[C]{\thepage} % 底部中央页码
%\renewcommand{\headrulewidth}{0pt} % 不显示页眉线
%\renewcommand{\footrulewidth}{0pt} % 不显示页脚线

%以下编辑为Final report设置
%%正文格式定义
\setlength{\parskip}{5pt plus 1pt minus 1pt} % space between paragraphs
\setlength{\parindent}{0pt} % paragraph indentation




%% 章节标题定义
\usepackage{titlesec}
\titleformat{\section}
{\fontsize{16}{18.4}\selectfont\bfseries\fontspec{Arial}}{\thesection}{0.5em}{}
\titleformat{\subsection}
{\fontsize{14}{16.1}\selectfont\bfseries\itshape\fontspec{Cambria}}{\thesubsection}{3em}{}
\titleformat{\subsubsection}
{\fontsize{12}{15}\selectfont\bfseries\itshape\fontspec{Cambria}}{\thesubsubsection}{3em}{}
\titlespacing*{\section}
{0pt}{12pt plus 4pt minus 2pt}{7pt plus 2pt minus 2pt}
\titlespacing*{\subsection}
{0.63cm}{12pt plus 4pt minus 2pt}{7pt plus 2pt minus 2pt}
\titlespacing*{\subsubsection}
{0.63cm}{12pt plus 4pt minus 2pt}{7pt plus 2pt minus 2pt}

%%目录
% 标题格式
\renewcommand{\cfttoctitlefont}{\bfseries\fontsize{16}{19}\selectfont\Arial}
\renewcommand{\cftaftertoctitle}{\vspace{12pt}\hspace*{\fill}}
% 目录项格式
\renewcommand{\cftsecfont}{\fontsize{11}{13}\selectfont}
\renewcommand{\cftsecpagefont}{\fontsize{11}{1}\selectfont}
\renewcommand{\cftsubsubsecfont}{\fontsize{11}{13}\selectfont}
\renewcommand{\cftsubsecindent}{0.39cm}
\renewcommand{\cftsubsubsecindent}{0.8cm} 
% 调整目录行距
% 调整目录行距和段前后距
\setlength{\cftbeforesecskip}{10pt}
\setlength{\cftbeforesubsecskip}{10pt}
\setlength{\cftbeforesubsubsecskip}{10pt}
\setlength{\cftbeforetoctitleskip}{0pt}
\setlength{\cftaftertoctitleskip}{10pt}
% 重命名
\renewcommand{\contentsname}{Table of Contents}

%%
% 设置“References”标题的格式

% 设置参考文献标题 "References"
%\titleformat{\section}
%{\normalfont\bfseries\fontsize{18}{22}\selectfont\fontfamily{phv}\selectfont}
%{\thesection}{1em}{}
%\titlespacing*{\section}{0pt}{12pt}{12pt}
%
%% 设置参考文献列表的样式
%\renewcommand{\cftsecfont}{\bfseries\fontsize{18}{22}\selectfont\fontfamily{phv}\selectfont}
%\renewcommand{\cftsecpagefont}{\bfseries\fontsize{18}{22}\selectfont\fontfamily{phv}\selectfont}

